\section{Various Distributions of Statistics}
    There are various distributions of Statistics that are used commonly, such as \textbf{Normal distribution (a.k.a. Gaussian), Student-T distribution, $\chi^2$ distribution, etc}. In this section, we learn how to approach these distributions and get proper values from them including the p-value. To reach the section, go to \textbf{Stats/List Editor}. \textbf{F4} presents you with information about input data, whereas \textbf{F5} gives diverse options of distributions, \textbf{F6} determines true or false between hypotheses $H_0$ and $H_a$, and \textbf{F7} gives confidence interval of the given sample, with selected distribution and parameters.

\subsection{Normal Distribution}
    A Normal distribution (a.k.a. Gaussian) is the most common for the SRS(Simple Random Sample), where it is approximated from the Binomial distribution by the Central Limit Theorem. There are two parameters, $\mu$ and $\sigma$, which represent the mean and standard deviation of the data set, respectively.
    
    \begin{equation}
        X \sim \mathcal{N}(\mu,\,\sigma^{2}) : f(x) = {1\over{\sigma \sqrt{2\pi}}} \exp{-{{(x-\mu)^2}\over{2\sigma^2}}}
        \label{def_Normal_Distribution}
    \end{equation}

    The graph of the distribution is symmetric, and bell-shaped, with the inflection points at $x=\mu \pm \sigma$. The standard normal distribution is more commonly used, expressed as $\mathcal{N}(0, 1)$. If probability variable X follows $N(\mu, \sigma^{2})$, it can be standardized to Z as

    \begin{equation}
        X \sim \mathcal{N}(\mu,\,\sigma^{2}) : Z = \frac{X-\mu}{\sigma} \sim\mathcal{N}(0, 1)
        \label{def_Standard_Normal_Distribution}
    \end{equation}

    Because of the standardization, the position of $X$ from $N(\mu, \sigma)$ is generalized to $Z$ from $N(0, 1)$ which is easily read from the calculator or \href{https://www.z-table.com/}{Z-score table}.
    
    \begin{itemize}
        \item NormalPdf\\
        
        Press \textbf{F5 3}.

        \item NormalCdf\\
        Returns the percentile of the interval from the normal distribution with given parameters. Type in lower bound, upper bound, $\mu$, $\sigma^{2}$.
        \begin{center}
            $Cdf = P(x_{lb}\leq X \leq x_{ub}),  X\sim \mathcal{N}(\mu,\,\sigma^{2})$
            \label{NormalCdf}
        \end{center}

        Press \textbf{F5 4}.

        \item Z-Test\\

        Press \textbf{F6 1}

        \item 2-SampZTest\\

        Press $F6$ $3$

        \item 1-PropZTest\\

        Press $F6$ $5$

        \item 2-PropZTest\\

        Press $F6$ 6

        \item Z-Interval\\
        This one might be important!

        Press $F7$ $1$

        \item 2-SampZInterval\\

        Press $F7$ $2$

        \item 1-PropZInterval\\

        Press $F7$ $5$

        \item 2-PropZInterval\\

        Press $F7$ $6$

        \item
        
    \end{itemize}

    For further information, take a look at \href{https://en.wikipedia.org/wiki/Normal_distribution}{Wikipedia}

\subsection{Student-T Distribution}
    A Normal distribution (a.k.a. Gaussian) is the most common for the SRS(Simple Random Sample), where it is approximated from the Binomial distribution by the Central Limit Theorem.

    \begin{itemize}
        \item tPdf\\
        Press \textbf{F5 3}.

        \item tCdf\\
        Returns the percentile of the interval from the normal distribution with given parameters. Input $LowVal$, $UpVal$, $\mu$, $\sigma^{2}$.

        Press \textbf{F5 4}.

        \item T-Test\\

        Press \textbf{F6 1}

        \item 2-SampTTest\\

        Press \textbf{F6 3}

        \item T-Interval\\

        Press \textbf{F7 1}

        \item 2-SampTInterval\\

        Press \textbf{F7 2}

    \end{itemize}

    For further information, take a look at \href{https://en.wikipedia.org/wiki/Student%27s_t-distribution}{Wikipedia}

\subsection{$\chi^{2}$ Distribution}
    A $\chi^{2}$ distribution is used for determining the independence of two variables, checking the homogeneity of the proportions, or judging whether a given linear fit is valid with the data set.
    
    \begin{equation}
        
        \label{def_Normal_Distribution}
    \end{equation}

    \begin{itemize}
        \item Chi-squarePdf\\
        
        Press \textbf{F5 3}.

        \item Chi-squareCdf\\
        Returns the percentile of the interval from the normal distribution with given parameters. Input $LowVal$, $UpVal$, $\mu$, $\sigma^{2}$.
        \begin{center}
            $Cdf = P(LowVal\leq X \leq UpVal),  X\sim \mathcal{N}(\mu,\,\sigma^{2})$
            \label{NormalCdf}
        \end{center}

        Press \textbf{F5 4}.

        \item Chi2 GOF\\

        Press $F6$ $1$

        \item Chi2 2-way\\

        Press $F6$ $3$

    \end{itemize}

\subsection{F Distribution}
\subsection{Binomial Distribution}
    Binomial Distribution is a discrete probability distribution of the number of successes from n independent trials that give boolean results: success, fail. The probability of success from a single trial is $p$, and the probability of failure is $q$. Each trial is called the Bernoulli experiment, and the distribution includes two parameters: $n$, $p$. The probability variable X is the number of successes; its distribution is expressed as

    \begin{equation}
        X \sim \mathcal{B}(n, p) : P(x = k) = {}_n C_k p^{k}q^{n-k}
        \label{def_Binomial_Distribution}
    \end{equation}

    The binomial distribution is related to the proportion since $p$ is the probability of success, but also it means the proportion of some proper set. By the Central Limit Theorem, the binomial can be approximated to the normal distribution when $np\geq5$ and $nq\geq5$.

    \begin{equation}
        E(X) = np, V(X) = npq
        \label{E(X), V(X): X ~ B(n, p)}
    \end{equation}

    \begin{equation}
        X\sim\mathcal{B}(n, p) : X \sim\mathcal{N}(np, npq)
        \label{B(n, p) -> N(np, npq)}
    \end{equation}

    \begin{itemize}
        \item BinomialCdf
        \item BinomialPdf   
    \end{itemize}

\subsection{Geometric Distribution}